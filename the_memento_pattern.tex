\documentclass[11pt, a4paper, twoside]{article}

%% Die folgende Anleitung gilt für Linux:
%%
%%  Die Übersetzung dieses Dokuments erfolgt mit (ggf. öfter aufrufen):
%%  pdflatex seminar.tex
%%
%%  Falls die Bibliographie geändert wurde, dann zusätzlich:
%%  bibtex seminar
%%
%%  Bitte darauf achten, dass pdflatex und bibtex keine Warnungen und
%%  keine Fehler melden.
%%
%%  Das Ergebnis kann man ansehen:
%%  acroread seminar.pdf
%%
%%  Die Quelldatei einer deutschen Rechtschreibprüfung unterziehen:
%%  ispell -d deutsch -T latin1 seminar.tex

%% Kodierung des Dokuments
\usepackage[utf8]{inputenc}

%% Titel und Autor der Ausarbeitung (BITTE ANPASSEN!)
\title{The Memento Design Pattern}
\author{Daniel Geier}
\date{} %% leer lassen

%% Grafiken einbetten
\usepackage{graphicx}

%% Verweise in der PDF
\usepackage{hyperref} % important: load after float
\hypersetup{
	pdfproducer={pdfeTex 3.14159-1.30.6-2.2},
	colorlinks=false,
	pdfborder=0 0 0	% keine Box um die Links!
}
\usepackage{url}
\urlstyle{rm}

%% Programmcode Listings
\usepackage{listings}
\lstset{basicstyle=\ttfamily}
\usepackage{color}

%% Seitenlayout (BITTE NICHT VERÄNDERN!)
\oddsidemargin0mm
\evensidemargin0mm
\textwidth159.2mm
\topmargin0mm
\headheight0mm
\textheight43\baselineskip

%% Header und Footer (BITTE NICHT VERÄNDERN!)
\usepackage{fancyhdr}
\pagestyle{fancy}
\fancypagestyle{plain}{\fancyhf{}\fancyfoot[RO,LE]{\thepage}\renewcommand\headrulewidth{0pt}}
\fancyhf{}
\fancyhead[LE]{\myauthor}
\fancyhead[RO]{\mytitle}
\fancyfoot[RO,LE]{\thepage}
\setlength{\headheight}{14pt}
\makeatletter
\let\mytitle\@title
\let\myauthor\@author
\makeatother

%\newcommand{\isdef}{=_{\mathrm{def}}}

% Der eigentliche Inhalt startet hier
\begin{document}
	\maketitle

	\begin{abstract} \noindent
		Memento is a design pattern used for saving and restoring the (partial) state of objects without breaking encapsulation. I give a comprehensive overview of Memento covering when and how to use it. Examples are provided by my own thing.
	\end{abstract}


	%%  Der Text der Ausarbeitung ist hier einzufügen.
	%%  Die Gliederung erfolgt mit \section und \subsection Befehlen.
	%%  Referenzen innerhalb des Texts erfolgen mit \label und \ref Befehlen.
	%%  Referenzen auf Literatur erfolgen mit dem \cite Befehl.
	\section{Motivation}
	\label{motivation}
	
	 
	 \section{Intent}
	
	\section{Structure}
	
	\section{Sample Code}
	
	\section{Discussion}
	
	\bibliographystyle{plain}
	\bibliography{the_memento_pattern}
\end{document}
